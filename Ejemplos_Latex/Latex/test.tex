\documentclass{article}

%\usepackage[utf8]{inputenc} % Paquete para aceptar caracteres en UTF-8
\usepackage[spanish]{babel} % Paquete para idioma español
\usepackage{amsmath, amsthm, amssymb}
\usepackage{graphicx}
\usepackage{blindtext}
%\graphicspath{{./img/}}
\usepackage{float}
\usepackage[table,xcdraw]{xcolor}

\title{Generar documento Latex con Visual Studio Code}
\author{Miguel Ángel Carrillo Lucía}
\date{\today}

\begin{document}

\maketitle

\section{Introducción}
\noindent
Aquí va el texto de tu introducción.
Aquí va algo más y \blindtext y sigo con más texto


\begin{equation}
    x=20
\end{equation}

Ahora se va a insertar una imagen

\begin{figure}[H]
    \centering
    \includegraphics[scale=0.2]{integral.png}
\end{figure}

Ya se probó con una imagen.

\section{Desarrollo}
Desarrollo de tus ideas principales. Ahora hay que intentar generar
una tabla. Se muestra a continuación:

% Please add the following required packages to your document preamble:
% \usepackage[table,xcdraw]{xcolor}
% Beamer presentation requires \usepackage{colortbl} instead of \usepackage[table,xcdraw]{xcolor}
\begin{table}[H]

    \centering
    \begin{tabular}{|c|c|c|c|c|}
    \hline
    \rowcolor[HTML]{CBCEFB} 
    \textbf{a} & \textbf{b} & \textbf{c} & \textbf{d} & \textbf{e} \\ \hline
    \rowcolor[HTML]{FFC702} 
    1          & 2          & 3          & 4          & 5          \\ \hline
    \rowcolor[HTML]{FFC702} 
    6          & 7          & 8          & 9          & 10         \\ \hline
    \rowcolor[HTML]{FFC702} 
    11         & 12         & 13         & 14         & 15         \\ \hline
    \end{tabular}
    \caption{Primera tabla generada en latex con visual studio code}
    \end{table}

\section{Conclusión}
Conclusión de tu documento.

\end{document}
