\documentclass{article}

\usepackage{geometry}
\geometry{left=2cm, 
right=3cm, 
top=.1cm,
bottom=3cm
 }

\usepackage{verbatim}
\title{Tarea 1}
\author{Miguel Ángel Carrillo Lucía \\ Leonardo David Solís Rodríguez}
\date{19 de agosto de 2025}

\begin{document}
\maketitle
\vspace{-10mm}

\section*{Entornos para el manejo de figuras en \LaTeX}

Elabore un documento en \textbf{Overleaf} en el cual se investiguen, comparen y ejemplifiquen los siguientes entornos:

\begin{itemize}
    \item \texttt{wrapfigure}
    \item \texttt{subfloat}
    \item \texttt{subfigure}
    \item \texttt{floatrow}
\end{itemize}

El documento debe cumplir con las siguientes indicaciones:

\begin{enumerate}
    \item \textbf{Investigación previa}
    \begin{itemize}
        \item Explique cómo se utiliza cada entorno.  
        \item Indique qué paquetes es necesario incluir para su funcionamiento.  
        \item Analice el caso del entorno \texttt{subfigure}: ¿por qué ya no se recomienda su uso frente a \texttt{subcaption}?  
    \end{itemize}
    
    \item \textbf{Ejemplos prácticos}
    \begin{itemize}
        \item Para \texttt{wrapfigure}, inserte dos imágenes acompañadas de un párrafo de texto que muestre cómo el contenido fluye alrededor de la figura. Una imagen debe alinearse a la izquierda y la otra a la derecha.  
        \item Para \texttt{subfloat}, utilice cuatro imágenes organizadas como subfiguras dentro de una figura principal.  
        \item Para \texttt{subcaption}, reproduzca el ejemplo revisado en clase y compárelo con los demás entornos.  
    \end{itemize}
    
    \item \textbf{Análisis comparativo}
    \begin{itemize}
        \item Compare las diferencias y similitudes entre los entornos estudiados.  
        \item Señale las ventajas y desventajas de cada uno.  
        \item Argumente en qué situaciones resulta más apropiado usar un entorno u otro.  
    \end{itemize}
    
    \item \textbf{Conclusiones}  
    Incluya una sección final con sus conclusiones personales.  
    
    \item \textbf{Aspectos de formato}
    \begin{itemize}
        \item Use los elementos de estilo vistos en clase: tipo de documento (\texttt{article}, \texttt{report}, \texttt{book}), tipo y tamaño de fuente, manejo de figuras y listas, color de texto, entre otros.  
        \item Organice el contenido en capítulos y secciones según corresponda.  
        \item Incluya una página independiente al final del documento con las \textbf{referencias consultadas}, en forma de lista enumerada.  
    \end{itemize}
\end{enumerate}

\textit{Sugerencia:} Para mostrar comandos sobre el texto, explore el paquete \verb|verbatim| y el comando asociado \verb|\verb|. \underline{No es necesario entregar informe sobre este paquete}.

\vspace{5mm}
\textbf{FECHA DE ENTREGA: LUNES 25 DE AGOSTO DE 2025.}

\end{document}